%%%%%%%%%%%%
%
% $Autor: Wings $
% $Datum: 2019-03-05 08:03:15Z $
% $Pfad: TemplateSensor $
% $Version: 4250 $
% !TeX spellcheck = en_GB/de_DE
% !TeX encoding = utf8
% !TeX root = filename 
% !TeX TXS-program:bibliography = txs:///biber
%
%%%%%%%%%%%%

% Structure
\chapter{OLED-Display }

In diesem Kapitel wird erklärt wie was ein OLED-Display ist, wie es funktioniert und wie es in diesem Projekt eingebunden wird.

\section{Allgemeine Beschreibung eines OLED-Displays}

Ein OLED-Display ist eine Art von Bildschirm, bei der organische Materialien Licht emittieren, wenn Strom durch sie hindurchfließt. "OLED" steht dabei für "Organic Light Emitting Diode".

Im Gegensatz zu LCD-Displays benötigen OLED-Displays keine Hintergrundbeleuchtung, da jede einzelne Pixelzelle selbst leuchtet. Außerdem sind OLED-Displays oft dünner, leichter, bieten eine besseren Kontrast und einen breiteren Betrachtungswinkel. 

Durch die schnelle Reaktionszeit von OLED-Displays werden sie vor allem in Smartphones, Fernsehern, Smartwatches oder High-End-Monitore verbaut. 

Nachteile von OLED-Displays liegen darin, dass sie teurer sind als LCD-Displays. Außerdem haben vor allem blaue OLED-Displays oft eine kürzere Lebensdauer und im Allgemeinen besteht die Gefahr des Einbrennens (Burn-in). Das bedeutet, dass es bei statischen Bildern dazu kommen kann, dass Reste des Bildes auf dem Display sichtbar bleiben. 

\section{Spezifische Beschreibung OLED-Displays}

Es wird ein 2,42 Zoll OLED-Diplay von Joy-IT verwendet. Dabei handelt es sich um ein Display mit 128 x 64 Pixeln, welches gelbe Schrift auf schwarzem Hintergrund darstellt. Es besitzt eine SPI- und eine I2C-Schnittstelle, sodass es mit dem verwendeten Arduino Nano 33 BLE Sense kompatibel ist. 

\section{Anschluss des Sensors mit dem Arduino Nano 33 BLE Sense}

In Tabelle xx ist gezeigt wie das OLED-Display korrekt an den Arduino Nano 33 BLE Sense angeschlossen wird:

\begin{table}[htpb]
	\centering
	\begin{tabular}{|l|c|}
		\hline
		\textbf{OLED-Display} & \textbf{Arudino Nano 33 BLE Sense} \\
		\hline
		1 & GND \\
		2 & 3V3 \\
		4 & D9 \\
		5 & D8 \\
		7 & A5 \\
		8 & A4 \\
		\hline
	\end{tabular}
	\caption{Pinbelegung für I2C-Verbindung zwischen OLED-Display und Arduino Nano 33 BLE Sense}
	\label{tab:PinOLED}
\end{table}

\section{Spezifikationen}

\begin{table}[htpb]
	\centering
	\begin{tabular}{|l|c|}
		\hline
		\textbf{Spezifikation} & \textbf{Wert} \\
		\hline
		Typ & Negativ OLED \\
		Auflösung & 128 x 64 Pixel \\
		Abmessungen & 71 x 50 x 7 mm \\
		Versorgungsspannung & 3 - 5 V \\
		Versorgungsstrom & 90 mA \\
		Logik Level & 3 V \\
		High Level Eingang & mind. 2,4 V \\
		Low Level Eingang & max. 0,6 V \\
		Helligkeit & 100 - 120 CD/$\mathrm{cm^2}$ \\
		Kontrast & > 2000:1 \\
		Blickwinkel & > 160° \\
		zuläss. Betriebstemperatur & -40 °C bis 85 °C \\
		Lagerungstemperatur & -45 °C bis 90 °C \\
		\hline
	\end{tabular}
	\caption{Spezifikationen des OLED-Displays}
	\label{tab:SpezifikationOLED}
\end{table}

\section{Bibliothek}

Für die Verwendung des OLED-Displays mit dem Arduino Nano 33 BLE Sense wird die U8g2 by oliver Bibliothek benötigt.

\textcolor{red}{Welche Funktionen werden verwendet + Erklärung}

\subsection{Installation}

Im Folgenden wird erklärt, wie die U8g2 Bibliothek installiert wird.

\begin{enumerate}
	\item Öffnen der Arduino IDE
	\item Klick auf Sketch, Bibliothek einbinden, Bibliotheken verwalten
	\item In der Suchleiste U8g2 eingeben
	\item Klick auf Installieren (Version aus Abbildung xx entnehmen)
\end{enumerate}

\begin{figure}[htpb]
	\centering
	\includegraphics[width=0.5\textwidth]{OLEDDisplay/U8g2-1.jpg}
	\caption{Installation der U8g2 Bibliothek}
	\label{fig:InstU8g2}
\end{figure}

\subsection{Code-Beispiel}

Auf Code-Beispiele für die U8g2 Bibliothek kann man über Datei, Beispiele, U8g2 zugreifen.

\section{Einfaches Beispiel mit Code}

Mit dem Code-Beispiel kann getestet werden, ob das Display korrekt angeschlossen ist. Wenn die Verbindung korrekt ist, dann erscheint auf dem Display "Hello World". 

\begin{figure}[htpb]
	\centering
	\includegraphics[width=1\textwidth]{OLEDDisplay/Code.jpg}
	\caption{wird noch in Code umgewandelt!}
	\label{fig:BeispielCodeDisplay}
\end{figure}

\section{Weiterführende Literatur}

\textcolor{red}{einfügen}

