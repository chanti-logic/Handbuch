%%%%%%%%%%%%%%%
%
% $Autor: Wings $
% $Datum: 2020-01-29 07:55:27Z $
% $Pfad: General/PowerTinyMLShield.tex
% $Version: 1785 $
%
%
%%%%%%%%%%%%%%%

%source: https://projecthub.arduino.cc/paulsb/temp-and-humidity-monitor-with-graphs-and-battery-monitor-cd011a

% https://www.az-delivery.de/products/az-delivery-laderegler-tp4056-mini-usb?variant=12239811084384
%https://www.az-delivery.de/products/cd60l-batterieladecontroller
%https://www.az-delivery.de/products/mini-solarpanel?variant=39475872792672

    
\chapter{Powering the TinyML Shield}

Now that you have your Arduino development board up and running, let's talk about how you could deploy it independent of your computer! While some embedded systems call on AC-DC converters (or “wall warts,” colloquially) to provide low voltage power to their electronics (the Google home speaker, for example), others are battery powered. Both of these paradigms are applicable to real-world deployment of tinyML and both are achievable using your TinyML kit.

\Mynote{Komplettes Kapitel muss überarbeitet werden. Sortierung, Quellen, Diagramm,\ldots}
\subsection{USB Power Delivery}

To this point, we have provided power over USB to our microcontroller via the microB port on the Nano 33 BLE Sense. The 5V that USB carries is then down regulated on the development board to 3.3V, the logical reference for the MCU. While there’s nothing wrong with this in development, a prototype of your application ought not depend on drawing power from your computer. Instead, you could call on an AC-DC converter with USB output. This has the added benefit of likely raising the current capacity of the 5V power rail, from at most 500 mA via a computer to whatever the specification happens to be for a given wall-bound converter. This could be meaningful for driving certain power hungry actuators, like speakers.

\section{Battery}

While the above solution removes the need for the computer, you’re still tethered to a wall. To go fully mobile you’ll want to call on a battery. So what are our options? The idea of using a voltage regulator (specifically, the MPM3610) to cut down an input voltage to a nice, stable 3.3V level applies here as well. If you were to take a closer look at the linked datasheet, you’d find that the MPM3610 accepts input voltages from 4.5V to 21V. The 5V delivered over USB is within this range, and any compatible battery will need to be as well. This unfortunately eliminates the possibility of calling on single cell 3.7V lithium batteries, but makes the selection of a 9V alkaline battery fairly obvious.

You might be wondering how any battery might connect to the boards in front of you, but never fear, we’ve got you covered. At one corner of the Tiny Machine Learning Shield you’ll find a green terminal block with silkscreen labels that read, “VIN” and “GND,” where GND is our reference voltage and as such should be connected to the negative terminal of any compatible battery. This green terminal block is where you’d want to screw in wires carrying 4.5V to 21V, and we’ll add that 9V clip, like this, that terminates in pre-stripped hook-up wire makes this quite easy!

\bigskip

Assembly Steps

\begin{enumerate}
    \item Screw down a wire leading from the negative battery terminal (black) to GND (Most < 3mm flat-head screwdrivers will suffice here).
    \item Repeat this process for the positive battery terminal (red) to VIN. And that's it you're all set to power your Arduino from a battery!
\end{enumerate}


\bigskip

Important Notes

\begin{enumerate}
    \item While there is clever circuitry on board to handle such an exception, it is generally good practice to avoid having competing power sources, so we’d recommend that you unplug the Nano 33 BLE Sense from USB power before connecting a battery
    
    \item With about 550 mAh capacity, a 9V battery can source 15 mA for about 37 hours before you will need to get a new one.
\end{enumerate}


\Mynote{tikz-Bild des Shields fehlt.}

\medskip

\begin{center}
  \includegraphics[width=\textwidth]{Battery/batteryScrew.png}
  \captionof{figure}{Montage des Clips}
\end{center}


An image of screwing down the black negative terminal to attach a wire.

\begin{center}
  \includegraphics[width=\textwidth]{Battery/batteryDone.png}
  \captionof{figure}{Komplettes System mit Batterie}
\end{center}

An image of the attached battery powering the device.


\section{Checking the Battery Voltage}

We use an analog input pin to read the voltage. As we are running from a 3.7V volt battery, we need to adjust the reference voltage used by the pin as otherwise it would be comparing the voltage to itself. The statement \PYTHON{analogReference(INTERNAL)} sets the pin to compare the input voltage to a regulated 1.1V. We therefore need to reduce the voltage on the input pin to less than 1.1V for this to work. This is done by dividing the voltage using 2 resistors, 1m and 330k ohms. This divides the voltage by approximately 4 so when the battery is fully charged, which is 4.2V, the voltage at the pin input is 4.2/4 = 1.05V. 

\begin{lstlisting}[language=python]
// Battery Monitor
#define MONITOR_PIN  A0              // Pin used to monitor supply voltage
const float voltageDivider  = 4.0;   // Used to calculate the actual voltage fRom the monitor pin reading
                                     // Using 1m and 330k ohm resistors divids the  voltage by approx 4
                                     // You may wany to substitute  actual values of resistors in an equation (R1 + R2)/R2
                                     // E.g. (1000 + 330)/330 = 4.03
                                     // Alternatively  take the voltage reading across the battery and from the joint between 
                                     //  the 2 resistors to ground and divide one by the other to get the value.
    
// Read the monitor pin and calculate the voltage 
float BatteryVoltage()
{ 
    float reading = analogRead(MONITOR_PIN); 
    // Calculate voltage - reference voltage is 1.1v 
    return 1.1 * (reading/1023) * voltageDivider; 
} 
\end{lstlisting}

The function \PYTHON{BatterVoltage()}, reads the analog pin, which will range from 0 for 0V to 1,023 for 1.1V and using this reading calculates the actual voltage coming form the battery. 

The function \PYTHON{DrawScreenSave()} function calls this then selects the appropriate bitmap to display based on the following: 

\begin{itemize}
    \item If voltage is greater then 3.6V - full 
    \item Voltage between 3.5 and 3.6V - 3/4 
    \item Voltage between 3.4 and 3.5V - half 
    \item Voltage between 3.3 and 3.4V - 1/4 
    \item Voltage < 3.3V - empty 
\end{itemize}



\section{Batterieclip}

Der Batterieclip in Abb. \ref{Batterieclip für 9-Volt-Block}, der vom Hersteller \textit{reichelt} ist, kann vertikal an einen 9-Volt-Block angeschlossen werden. Die dazugehörigen Anschlussdrähte haben eine Länge von 150 mm. Der Anschlussclip ist in der I-Form ausgeführt, weshalb er sich platzsparend ins Gehäuse einbinden lässt \cite{Reichelt:2011}.

\begin{figure}[h]
    \begin{center}
        \includegraphics[width=3in]{Battery/clip.jpg}
        \caption{Batterieclip für 9-Volt-Block\cite{Reichelt:2024a}}
        \label{Batterieclip für 9-Volt-Block}
    \end{center}
\end{figure} 



\section{Spannungssensor}

Der Spannungssensor in Abb. \ref{Spannungssensor} von dem Hersteller \textit{Shenzhen Global Technology Co., Ltd} kann bei der Versorgungsspannung von 3,3 V Spannungen in dem Bereich von 0 V bis 16,5 V messen. Dieser wird genutzt, um den Ladestand der Batterie zu überwachen. Die analoge Auflösung des Sensors liegt bei 10 Bit. Damit kann bei dem angegebenen Messbereich die Spannung mit der Auflösung von 0,016 V gemessen werden. Zur Eingangsschnittstelle gehört der \ac{vcc}-Anschluss und der \ac{gnd}-Anschluss \cite{Shenzhen:2015}. Die Bauteilmaße betragen 13 mm x 27 mm.



{\begin{center}
		
  \includegraphics[width=0.6\textwidth]{Battery/VoltageSensor}
  
  \captionof{figure}{Voltage Sensor 170640  von dem Hersteller \textit{Shenzhen Global Technology Co., Ltd}\cite{Shenzhen:2015}}   		
		
	\end{center}
}
	
Mit dem Sketch \FILE{TestBattery.ino}, siehe \ref~{TestBattery}, soll der Spannungssensor getestet werden.

\medskip



\begin{code}
	
	\caption{Beispiel-Sketch zum Testen der Batterie}\label{TestBattery}

    \ArduinoExternal{}{../../Code/Arduino/Battery/TestBattery.ino}
\end{code}

\subsection{Durchführung}

Für den Test werden die folgenden Hardware-Komponenten benötigt:

\begin{itemize}
    \item Arduino Nano 33 BLE Sense Lite
    \item Tiny Machine Learning Shield
    \item USB-A auf USB-Mikro Verbindungskabel
    \item Grove Jumper zu Grove 4 Pin Kabel
    \item Spannungssensor
    \item Batterie
    \item Batterieclip
\end{itemize}

Die Hardware-Komponenten werden  zusammengebaut, aber die Batterie wird noch nicht angeschlossen. Dann wird der Arduino Nano 33 BLE Sense Lite mit einem Computer verbunden. Anschließend wird der Sketch \FILE{TestBattery.ino} auf den Arduino Nano 33 BLE Sense Lite geladen und der serielle Monitor in der Arduino \acs{ide} geöffnet. Die Batterie wird während des Tests an den Batterieclip angeschlossen und der gemessene Wert bei dem seriellen Monitor ausgelesen.

\subsection{Ergebnisse}

Zu Beginn des Tests zeigt der serielle Monitor die Spannung 0 V an. Nach dem Anschließen der Batterie an den Spannungssensor wird die Spannung 9,6 V angezeigt.  Abbildung~\ref{BildSpannungTest} zeigt den gemessenen Spannungsverlauf. Die angezeigte Spannung liegt in dem erwarteten Bereich einer 9 V Batterie.

\begin{figure}[h]
    \begin{center}
        \includegraphics[width=\textwidth]{Battery/BatteryTest.png}
        \caption{Testoutput des Spannungssensors}
        \label{BildSpannungTest}
    \end{center}
\end{figure}

