%%%%%%%%%%%%%%%
%
% $Autor: Wings $
% $Datum: 2020-01-29 07:55:27Z $
% $Pfad: General/Battery.tex
% $Version: 1785 $
%
%
%%%%%%%%%%%%%%%

%source: https://projecthub.arduino.cc/paulsb/temp-and-humidity-monitor-with-graphs-and-battery-monitor-cd011a

% https://www.az-delivery.de/products/az-delivery-laderegler-tp4056-mini-usb?variant=12239811084384
%https://www.az-delivery.de/products/cd60l-batterieladecontroller
%https://www.az-delivery.de/products/mini-solarpanel?variant=39475872792672

\chapter{Hypatia  - Li-Po batteries, Pins and board LEDs}

\section{Battery capacity}
%source: MLbib/Arduino/MKR/GKX00006_ENG_TDS_241122_121923.pdf

Li-Po batteries are charged up to 4,2V with a current that is usually half of the nominal capacity (C/2). For Hypatia  we use a specialized chip that has a preset charging current of \textcolor{red}{350mAh}. This means that the \textbf{minimum} capacity of the Li-Po battery should be 700 mAh. Smaller cells will be damaged by this current and may overheat, develop internal gasses and explode, setting on fire the surroundings. We strongly recommend that you select a Li-Po battery of at least 700mAh capacity. A bigger cell will take more
time to charge, but won't be harmed or overheated. The chip is programmed with \textcolor{red}{4} hours of charging time, \textcolor{red}{then it goes into automatic sleep mode}. This will limit the amount of charge to maximal \textcolor{red}{1400 mAh} per charging round.

\section{Battery connector}

If you want to connect a battery to your Hypatia be sure to search one with female 2 pin JST PHR2 Type connector.

\medskip

Polarity : looking at the board connector pins, polarity is Left = Positive, Right = GNDDownload


\medskip

On the Hypatia, connector is a male 2 pin JST PH Type  Vin. This pin can be
used to power the board with a regulated 5V source. If the power is fed through this pin.

If  USB power source is connected, then the battery is not in use. The onboard energy management  checks whether the battery is full. If the battery is not full, it will be loaded.

 This is the only way you can supply 5V (range is 5V to maximum 6V) to the board not using USB.
  This pin is an INPUT. \Mynote{which pin}
  
\section{5V}

This pin outputs 5V from the the board when powered from the USB connector or from the  pin VIN of the board. It is unregulated and the voltage is taken directly from the inputs. When powered from battery it supplies around 3.7 V. As an OUTPUT, it should not be used as an input pin to power the board.

\section{VCC}

This pin outputs 3.3V through the on-board voltage regulator. This voltage is the same regardless the power
source used (USB, Vin and Battery).

\section{LED ON}

This LED is connected to the 5V input from either USB or VIN. It is not connected to the battery power. This means that it lits up when power is from USB or VIN, but stays off when the board is running on battery power. This maximizes the usage of the energy stored in the battery. It is therefore normal to have the board properly running on battery power without the LED ON being lit.

\section{CHARGE LED}

The CHARGE LED on the board is driven by the charger chip that monitors the current drawn by the Li-Po battery while charging. Usually it will lit up when the board gets 5V from VIN or USB and the chip starts charging the Li-Po battery connected to the JST connector.There are several occasions where this LED will start to blink at a frequency of about 2Hz. This flashing is caused by the following conditions maintained for a long time (from 20 to 70 minutes):

\begin{itemize}
    \item  No battery is connected to JST connector.
    \item  Overdischarged/damaged battery is connected. It can't be recharged.
    \item A fully charged battery is put through another unnecessary charging cycle. This is done disconnecting and reconnecting either VIN or the battery itself while VIN is connected.
\end{itemize}

\section{Onboard LED}

On MKR1000 the onboard LED is connected to D6 and not D13 as on the other boards. Blink example or other sketcthes that uses pin 13 for onboard LED may need to be changed to work properly.
    

\section{Pin Assignment}

The voltage of the battery is connected to pin 10.\index{Pin!Pin 10} The actual voltage can not be measured. 

\begin{description}
    \item [Built-in Battery Voltage:] \PYTHON{PinBatteryVoltage =  10u}
\end{description}

The Voltage is analogue value with 10 bits.


The voltage of the battery can be measured by reading the pin: \PYTHON{analogRead(PinBatteryVoltage)}.



%The pin 10 must be defined as an input in the function \PYTHON{setup} by setting \PYTHON{pinMode (PinBatteryVoltage, INPUT)}.


\section{Checking the Battery Voltage with Hypatia}


 
%\begin{code}
%    \ArduinoExternal{../../Code/Syde/Test/TestBattery/TestBattery.ino}
%\end{code}

\section{Battery Example}

\URL{https://ampul.eu/it/batterie/4079-batteria-li-pol-8000mah-37v-126090}

Batteria Li-Pol 8000mAh, 3,7V, 126090

Wiederaufladbare Lithium-Polymer-Akkus werden in allen elektronischen Geräten für den Privatgebrauch verwendet (z. B. Mobiltelefone, Kameras, Laptops, RC-Modelle, ...). Es ist nicht notwendig, den Akku vor dem Aufladen vollständig zu entladen, es gibt keinen Memory-Effekt. Einfache und nahtlose Verbindung vieler Zellen in Reihe.

\bigskip


Integrierter Schutzchip.

Spannung: 3,7V

Kapazität: 8000mAh

Ladespannung: 3,7-4,2V DC

2000 Ladezyklen

Minimale Selbstentladung (ca. 5% pro Monat)

Betriebstemperatur: -10 - 50°C

Abmessungen: 90x60x12 mm

Typenbezeichnung: 126090

WARNUNG: Fällt die Spannung unter 2,7 V, kann die Batterie irreversibel zerstört werden. 
    
    
\chapter{Battery}

\section{Appendix - Powering Your TinyML System}

Now that you have your Arduino development board up and running, let's talk about how you could deploy it independent of your computer! While some embedded systems call on AC-DC converters (or “wall warts,” colloquially) to provide low voltage power to their electronics (the Google home speaker, for example), others are battery powered. Both of these paradigms are applicable to real-world deployment of tinyML and both are achievable using your TinyML kit.

\subsection{USB Power Delivery}

To this point, we have provided power over USB to our microcontroller via the microB port on the Nano 33 BLE Sense. The 5V that USB carries is then down regulated on the development board to 3.3V, the logical reference for the MCU. While there’s nothing wrong with this in development, a prototype of your application ought not depend on drawing power from your computer. Instead, you could call on an AC-DC converter with USB output. This has the added benefit of likely raising the current capacity of the 5V power rail, from at most 500 mA via a computer to whatever the specification happens to be for a given wall-bound converter. This could be meaningful for driving certain power hungry actuators, like speakers.

\section{Battery}

While the above solution removes the need for the computer, you’re still tethered to a wall. To go fully mobile you’ll want to call on a battery. So what are our options? The idea of using a voltage regulator (specifically, the MPM3610) to cut down an input voltage to a nice, stable 3.3V level applies here as well. If you were to take a closer look at the linked datasheet, you’d find that the MPM3610 accepts input voltages from 4.5V to 21V. The 5V delivered over USB is within this range, and any compatible battery will need to be as well. This unfortunately eliminates the possibility of calling on single cell 3.7V lithium batteries, but makes the selection of a 9V alkaline battery fairly obvious.

You might be wondering how any battery might connect to the boards in front of you, but never fear, we’ve got you covered. At one corner of the Tiny Machine Learning Shield you’ll find a green terminal block with silkscreen labels that read, “VIN” and “GND,” where GND is our reference voltage and as such should be connected to the negative terminal of any compatible battery. This green terminal block is where you’d want to screw in wires carrying 4.5V to 21V, and we’ll add that 9V clip, like this, that terminates in pre-stripped hook-up wire makes this quite easy!

\bigskip

Assembly Steps

\begin{enumerate}
    \item Screw down a wire leading from the negative battery terminal (black) to GND (Most < 3mm flat-head screwdrivers will suffice here).
    \item Repeat this process for the positive battery terminal (red) to VIN. And that's it you're all set to power your Arduino from a battery!
\end{enumerate}


\bigskip

Important Notes

\begin{enumerate}
    \item While there is clever circuitry on board to handle such an exception, it is generally good practice to avoid having competing power sources, so we’d recommend that you unplug the Nano 33 BLE Sense from USB power before connecting a battery
    
    \item With about 550 mAh capacity, a 9V battery can source 15 mA for about 37 hours before you will need to get a new one.
\end{enumerate}

\includegraphics[width=\textwidth]{Battery/batteryScrew.png}


An image of screwing down the black negative terminal to attach a wire.

\includegraphics[width=\textwidth]{Battery/batteryDone.png}

An image of the attached battery powering the device.


\section{Checking the Battery Voltage}

We use an analog input pin to read the voltage. As we are running from a 3.7V volt battery, we need to adjust the reference voltage used by the pin as otherwise it would be comparing the voltage to itself. The statement \PYTHON{analogReference(INTERNAL)} sets the pin to compare the input voltage to a regulated 1.1V. We therefore need to reduce the voltage on the input pin to less than 1.1V for this to work. This is done by dividing the voltage using 2 resistors, 1m and 330k ohms. This divides the voltage by approximately 4 so when the battery is fully charged, which is 4.2V, the voltage at the pin input is 4.2/4 = 1.05V. 

\begin{lstlisting}[language=python]
// Battery Monitor
#define MONITOR_PIN  A0              // Pin used to monitor supply voltage
const float voltageDivider  = 4.0;   // Used to calculate the actual voltage fRom the monitor pin reading
                                     // Using 1m and 330k ohm resistors divids the  voltage by approx 4
                                     // You may wany to substitute  actual values of resistors in an equation (R1 + R2)/R2
                                     // E.g. (1000 + 330)/330 = 4.03
                                     // Alternatively  take the voltage reading across the battery and from the joint between 
                                     //  the 2 resistors to ground and divide one by the other to get the value.
    
// Read the monitor pin and calculate the voltage 
float BatteryVoltage()
{ 
    float reading = analogRead(MONITOR_PIN); 
    // Calculate voltage - reference voltage is 1.1v 
    return 1.1 * (reading/1023) * voltageDivider; 
} 
\end{lstlisting}

The function \PYTHON{BatterVoltage()}, reads the analog pin, which will range from 0 for 0V to 1,023 for 1.1V and using this reading calculates the actual voltage coming form the battery. 

The function \PYTHON{DrawScreenSave()} function calls this then selects the appropriate bitmap to display based on the following: 

\begin{itemize}
    \item If voltage is greater then 3.6V - full 
    \item Voltage between 3.5 and 3.6V - 3/4 
    \item Voltage between 3.4 and 3.5V - half 
    \item Voltage between 3.3 and 3.4V - 1/4 
    \item Voltage < 3.3V - empty 
\end{itemize}



\section{Batterieclip}
Der Batterieclip in Abb. \ref{Batterieclip für 9-Volt-Block}, der vom Hersteller \textit{reichelt} ist, kann vertikal an einen 9-Volt-Block angeschlossen werden. Die dazugehörigen Anschlussdrähte haben eine Länge von 150 mm. Der Anschlussclip ist in der I-Form ausgeführt, weshalb er sich platzsparend ins Gehäuse einbinden lässt \cite{Reichelt:2011}.

\begin{figure}[h]
    \begin{center}
        \includegraphics[width=3in]{Battery/clip.jpg}
        \caption{Batterieclip für 9-Volt-Block\cite{Reichelt:2024a}}
        \label{Batterieclip für 9-Volt-Block}
    \end{center}
\end{figure} 

\section{Spannungssensor}
Der Spannungssensor in Abb. \ref{Spannungssensor} von dem Hersteller \textit{Shenzhen Global Technology Co., Ltd} kann bei der Versorgungsspannung von 3,3 V Spannungen in dem Bereich von 0 V bis 16,5 V messen. Dieser wird genutzt, um den Ladestand der Batterie zu überwachen. Die analoge Auflösung des Sensors liegt bei 10 Bit. Damit kann bei dem angegebenen Messbereich die Spannung mit der Auflösung von 0,016 V gemessen werden. Zur Eingangsschnittstelle gehört der \ac{vcc}-Anschluss und der \ac{gnd}-Anschluss \cite{Shenzhen:2015}. Die Bauteilmaße betragen 13 mm x 27 mm.

\section{Spannungssensor}
Mit dem Sketch \FILE{TestBattery.ino} soll der Spannungssensor getestet werden.

\begin{code}[h]
    \pythonexternal[language=c++]{../../Code/Arduino/Battery/TestBattery.ino}
\end{code}

\subsection{Durchführung}

Für den Test werden die folgenden Hardware-Komponenten benötigt:

\begin{itemize}
    \item Arduino Nano 33 BLE Sense Lite
    \item Tiny Machine Learning Shield
    \item USB-A auf USB-Mikro Verbindungskabel
    \item Grove Jumper zu Grove 4 Pin Kabel
    \item Spannungssensor
    \item Batterie
    \item Batterieclip
\end{itemize}

Die Hardware-Komponenten werden  zusammengebaut, aber die Batterie wird noch nicht angeschlossen. Dann wird der Arduino Nano 33 BLE Sense Lite mit einem Computer verbunden. Anschließend wird der Sketch \FILE{TestBattery.ino} auf den Arduino Nano 33 BLE Sense Lite geladen und der serielle Monitor in der Arduino \acs{ide} geöffnet. Die Batterie wird während des Tests an den Batterieclip angeschlossen und der gemessene Wert bei dem seriellen Monitor ausgelesen.

\subsection{Ergebnisse}

Zu Beginn des Tests zeigt der serielle Monitor die Spannung 0 V an. Nach dem Anschließen der Batterie an den Spannungssensor wird die Spannung 9,6 V angezeigt.  Abbildung~\ref{BildSpannungTest} zeigt den gemessenen Spannungsverlauf. Die angezeigte Spannung liegt in dem erwarteten Bereich einer 9 V Batterie.


\begin{center}
  \includegraphics[width=\textwidth]{Battery/BatteryTest.png}
  \captionof{figure}{Testoutput des Spannungssensors}
  \label{BildSpannungTest}
\end{center}

\section{Lithium Battery}


Betrieben wird der Arduino Nano 33 BLE Sense mit einer Lithium Batterie CR2032: \cite{Energizer:2018}
    
    \URL{https://data.energizer.com/pdfs/cr2032.pdf}           
    

\subsection{Attention}


Der Arduino Nano 33 BLE Sense lässt sich über die eingebaute Micro-USB Schnittstelle programmieren.

Dazu muss  der Jumper geschlossen (gesteckt werden).


\textbf{ACHTUNG vor dem Stecken des Jumpers muss der Stab ausgeschaltet sein und er darf
    nicht eingeschaltet werden, solange der Jumper steckt!!}


\begin{center}
    
  \includegraphics[width=0.8\textwidth]{MagicWand/MagicWandS}
  \captionof{figure}{Example with Battery and Jumper}
   \label{Figure:jumperconnection}
\end{center}







\subsection{Battery}

The battery of the Arduino Nano 33 BLE Sense is a button battery CR2032 on Lithium, to used correctly the battery we need a specific connection. This battery connection is also connect on Power button. 

The battery need some description about the capacity, the current voltage for Arduino Nano 33 BLE Sense battery is about 3.3V. The button cell batteries can have a capacity of up to 3.7V, making them ideal for the Arduino Nano, since they are only used occasionally. The voltage will then vary between 3V when it's full and 2.6V when it's empty. 


The connection between battery and card is something particular with the power button, we connect the battery and the card like this :  

\begin{itemize}
    \item The positive part of the battery is connected to one part of the button ON/OFF.
    \item The second part of the button ON/OFF button is connected to the +3.3V battery. 
    \item The negative go to Arduino GND to button battery GND. 
\end{itemize}


\subsection{Jumper}

The jumper consists of two wires that never touch. When the jumper is not activated, either the two wires are touching, or there is no soldering and the jumper is not cut; this case is effective when there is no battery. 

When a battery is added, the jumper is set \ref{Figure:JumperPhoto}, preventing the creation of a short circuit.

The battery's GND is on the board's GND (in black) and the battery's 3V is on the board's 3.3V. The jumper, on the other hand, is ``in a vacuum'', and the battery and jumper connections are defined by the board below \ref{Figure:jumperconnection}. 

The creation and installation of this jumper is essential to the operation of the battery. Without the jumper in place, the arduino board would be in danger. 



\begin{center}
    \includegraphics[width=4cm]{MagicWand/jumper}
    \captionof{figure}{Jumper connection Arduino card}
    \label{Figure:jumperconnection}
    
\Mynote{tikz picture with numbers}    

    \bigskip
    
    \includegraphics[width=5cm]{MagicWand/jumperphoto}
    \captionof{figure}{Photo of the jumper on the project}
    \label{Figure:JumperPhoto}
\end{center} 


\URL{https://forum.arduino.cc/t/can-i-power-the-nano-33-ble-via-the-3v3-pin-with-3-3v/614634}


I tried both ways to power nano 33 ble and as you can imagine powering from Vin is less efficient than from 3.3v pin. To be more specific for a basic sketch where nano acts like a peripheral and sends gyro + accel values to a central device, powering from vin through a 5v booster the power consumption was ~18mAh. The same sketch consumes ~9.30mAh powering from 3.3v pin without any low power configuration. Just have in mind that in order to use direct the 3.3v pin you have to cut out the relative jumper on the bottom layer. Also, if you want to re flush nano board you have to connect the jumper again, otherwise usb connection does not work. I suppose for that you can do something with the Vusb jumper but i did not try it yet.


So as if i understand correctly I can wire 3.3V to the 3V3 pin under the condition:

Just have in mind that in order to use direct the 3.3v pin you have to cut out the relative jumper on the bottom layer.

Which means I have to break this connection marked with the blue square?

[EDIT: While doing more research on this I could not find any information on the SJ4 connection on the internet. The the only thing I was able to find and is mentioned everywhere is information about SJ1. Could you give me the source of the information about SJ4? What is the orientation of these pads? I am using the references from the official datasheet.]

\URL{https://content.arduino.cc/assets/NANO33BLE_V2.0_sch.pdf}


\URL{https://forum.arduino.cc/t/power-nano-33-ble-sense-with-3-3v/654217}

\URL{https://dumblebots.com/2020/03/27/getting-started-with-arduino-nano-33-ble-and-ble-sense/}

\URL{https://community.element14.com/products/roadtest/b/blog/posts/review-of-development-board-arduino-nano-33-ble-sense-roadtest}

\URL{https://tinyml.seas.harvard.edu/assets/other/4D/22.03.11_Marcelo_Rovai_Handout.pdf}

\URL{https://learn.adafruit.com/circuitpython-on-the-arduino-nano-rp2040-connect?view=all}

\URL{https://support.arduino.cc/hc/en-us/articles/360014735580-Nano-boards-that-can-be-powered-directly-with-3-3-V}
