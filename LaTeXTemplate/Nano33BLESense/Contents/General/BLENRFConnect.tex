%%%%%%
%
% $Author: Sandesh Nonavinakere Sunil $
% $Date: today $
% $Pfad: manual/Contents/en/Gettingstarted.tex $
% $Version: 1 $
%
%%%%%%



\chapter{Getting Started}
\label{chap:list_of_items}


\section{Download and Install the NRF Connect App on Your Mobile}

To download and install the \textbf{nRF Connect} app on Android or iOS devices, follow these simple steps:

\subsection{For Android}

\begin{itemize}
	\item Open the \textbf{Google Play Store} on your Android device.
	\item In the search bar, type \textit{"nRF Connect for Mobile"}.
	\item Select the app by \textbf{Nordic Semiconductor ASA}.
	\item Tap the button \textbf{``Install''} to download and install the app.
	\item Once installed, you can open the app from your home screen or app drawer.
\end{itemize}


\subsection{For iOS}

\begin{itemize}
	\item Open the \textbf{App Store} on your iOS device.
	\item In the search bar, type \textit{``nRF Connect for Mobile''}.
	\item Select the app by \textbf{Nordic Semiconductor ASA}.
	\item Tap the button \textbf{``Get''} to download and install the app.
	\item Once installed, you can open the app from your home screen or app drawer.
\end{itemize}\noindent The \textbf{nRF Connect} app allows you to interact with \textbf{Bluetooth Low Energy (BLE)} devices, such as the Arduino Nano 33 BLE, and monitor their services and data.

\begin{center}
	\includegraphics[width=0.5\textwidth]{BLE/NRFConnect}
	\captionof{figure}{NRF Connect Application}
	\label{fig:NRFConnectDownloadPage}
\end{center}

\section{Power ON the Arduino Nano 33 BLE Sense Lite}
To power on the Arduino Nano 33 BLE, connect it to your computer or a USB power adapter using the micro-USB cable provided in the box.

\subsection{Connect to Arduino Nano 33 BLE Sense Lite through NRFconnect}

\subsection{For Android}
\begin{enumerate}[label=\textbullet]
	\item \textbf{Open the NRFconnect App}: Open the app and start scanning for nearby BLE devices.
	\item \textbf{Find The Arduino}: You should see a device named \textbf{``MyProctName''} in the list of available devices.
	\item \textbf{Connect}: Tap on the ArduinoNano33 device to connect. The Arduino Nano will now show ``Connected'' in the Mobile application.
\end{enumerate}

\subsection{For iOS}
\begin{enumerate}[label=\textbullet]
	\item \textbf{Open the NRFconnect App}: Open the app and start scanning for nearby BLE devices.
	\item \textbf{Find Your Arduino}: Look for \textbf{``MyProctName''} in the list of detected BLE devices.
	\item \textbf{Connect}: Tap the device name to connect. You will be able to see connection status in the Mobile application.
\end{enumerate}

\begin{center}
	%\begin{subfigure}[b]{0.45\textwidth} % First image (45% of text width)
	%\centering
	\includegraphics[width=0.45\textwidth]{BLE/BlueToothIcon.jpg}
	\captionof{figure}{Landing page}
	\label{fig:Landing page of App}
\end{center}

\bigskip

\begin{center}
	%\begin{subfigure}[b]{0.45\textwidth} % Second image (45% of text width)

	\includegraphics[width=0.45\textwidth]{BLE/ListOfDevices} % Replace with your second image
	\captionof{figure}{List of devices}
	\label{fig:List of devices}
\end{center}



