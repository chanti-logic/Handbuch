%%%%%%%%%%%%
%
% $Autor: Wings $
% $Datum: 2019-03-05 08:03:15Z $
% $Pfad: TemplateSensor $
% $Version: 4250 $
% !TeX spellcheck = en_GB/de_DE
% !TeX encoding = utf8
% !TeX root = filename 
% !TeX TXS-program:bibliography = txs:///biber
%
%%%%%%%%%%%%

% Structure
\chapter{Schrittmotortreiber DRV8825}

In diesem Kapitel wird beschrieben, was ein Motortreiber ist, welche Aufgaben er erfüllt und wie man den DRV8825 mit dem Arduino Nano 33 BLE Sense verwendet.

\section{Allgemeine Beschreibung eines Motortreibers}

Ein Motortreiber ist ein Elektronikbaustein, der Mikrocontrollern wie dem Arduino Nano 33 BLE Sense ermöglicht elektrische Motoren anzusteuern. Dabei kann es sich zum Beispiel um Gleichstrommotoren oder Schrittmotoren handeln. 

Motortreiber haben mehrere Funktionen. Zum einen verstärken sie die Spannung und den Strom, da Mikrocontroller meist zu wenig Leistung liefern, um den Motor direkt zu betreiben. Zum anderen sorgen Motortreiber dafür, dass man die Drehrichtung und die Drehzahl regeln kann. Außerdem haben viele Motortreiber auch eine Schutzfunktion, die vor Überstrom, Überhitzung oder Kurzschluss schützen.


\section{Spezifische Beschreibung des Schrittmotortreibers DRV8825}

Der DRV8825 ist ein Motortreiber, der vor für bipolare Schrittmotoren wie hier der NEMA 17-04 eingesetzt wird. Er wird in Druckern, CNC-Maschinen oder in der Robotik verwendet.
 
Der Motor lässt sich über den DRV8825 von Vollschritt bis 1/32-Schritt einstellen. Er verfügt über einen konfigurierbaren Abklingmodus (tritt bei Erreichen des Stromlimits ein), sodass langsames, schnelles oder gemischtes Abklingen verwendet werden kann. Außerdem besitzt er einen stromsparenden Schlafmodus, der die interne Schaltung abschaltet und den Ruhestromverbrauch minimiert. 

Der DRV8825 verfügt über Schutzfunktionen für Überstrom, Kurzschluss, Unterspannung und Übertemperatur. Außerdem werden Fehler über den nFAULT-Pin (siehe Abbildung xx) signalisiert. 

\subsection{Anschluss des Treibers mit dem Arduino Nano 33 BLE Sense}

Tabelle xx kann entnommen werde, wie der DRV8825 an den Arduino Nano 33 BLE Sense angeschlossen wird. Tabelle xx zeigt, wie der DRV8825 an den Schrittmotor NEMA 17-04 angeschlossen wird. 

\begin{table}[htpb]
	\centering
	\begin{tabular}{|l|c|}
		\hline
		\textbf{DRV8825} & \textbf{Arduino Nano 33 BLE Sense} \\
		\hline
		STEP & z.B. D3 \\
		DIR & z.B. D4 \\
		EN & GND \\
		nRESET & HIGH (z.B. an 5V) \\
		nSLEEP & HIGH (z.B. an 5V) \\
		GND & GND \\
		VDD & 5V \\
		\hline
	\end{tabular}
	\caption{Pin-Belegung für den Anschluss des Treibers an den Arduino}
	\label{tab:PinDRV8825-Arduino}
\end{table}

\begin{table}[htpb]
	\centering
	\begin{tabular}{|l|c|}
		\hline
		\textbf{DRV8825} & \textbf{Schrittmotor} \\
		\hline
		AOUT1 & Spule A, Ende 1 \\
		AOUT2 & Spule A, Ende 2 \\
		BOUT1 & Spule B, Ende 1 \\
		BOUT2 & Spule B, Ende 2 \\
		\hline
	\end{tabular}
	\caption{Pin-Belegung für den Anschluss des Treibers an den Schrittmotor}
	\label{tab:PinDRV8825-Motor}
\end{table}

\section{Spezifikationen}

Die Spezifikationen sind in Tabelle xx und xx zu sehen.

\begin{table}[htpb]
	\centering
	\begin{tabular}{|l|c|}
		\hline
		\textbf{Spezifikation} & \textbf{Wert} \\
		\hline
		Versorgungsspannung & -0,3 V bis 47 V \\
		Digitale Eingangsspannung & -0,5 V bis 7 V \\
		Referenzspannung & -0,3 V bis 4 V \\
		Motorstrom & bis zu 2,5 A \\
		Temperaturbereich & -40 °C bis 150 °C \\
		\hline
	\end{tabular}
	\caption{Absolute Maximalwerte}
	\label{tab:maxSpezifikationDRV8825}
\end{table}

\begin{table}[htpb]
	\centering
	\begin{tabular}{|l|c|}
		\hline
		\textbf{Spezifikation} & \textbf{Wert} \\
		\hline
		Versorgungsspannung & 8,2 V - 45 V \\
		Referenzspannung & 1 V bis 3,5 V \\
		Motorstrom & 0 A bis 2,5 A \\
		\hline
	\end{tabular}
	\caption{Empfohlene Betriebsbedingungen}
	\label{tab:SpezifikationDRV8825}
\end{table}

\section{Kalibrierung}

Da der Treiber den Strom durch den Schrittmotor regelt ist die Kalibrierung des DRV8825 erforderlich, um zu vermeiden, dass der Schrittmotor überhitzt oder der Treiber überlastet. 

Für die Kalibrierung wird ein Multimeter, ein Schraubendreher für das Potentiometer, der GND-Pin des Arduino und der VREF-Messpunkt (Metallplättchen am Potentiometer) benötigt.

Im ersten Schritt ist es wichtig, dass der Schrittmotor nicht angeschlossen ist. Als nächstes wird die Stromversorgung eingeschaltet. Nun wird das Multimeter verwendet, dabei wird der Pluspol am VREF-Messpunkt verbunden und der Minuspol am GND-Anschluss des Arduino. Aus dem abgelesenen Wert kann nun die Zielspannung für VREF berechnet werden und dann am Potentiometer mit dem Schraubendreher eingestellt werden. Dabei ist darauf zu achten das Potentiometer nicht zu überdrehen, da es empfindlich ist. 

Nach dem Kalibrieren des Treibers kann der Schrittmotor angeschlossen werden. 
 

\section{Einfaches Beispiel mit Code}

\begin{figure}[htpb]
	\centering
	\includegraphics[width=1\textwidth]{DRV8825/Code.jpg}
	\label{fig:BeispielCodeDRV8825}
\end{figure}

\section{Weiterführende Literatur}

\textcolor{red}{einfügen}

